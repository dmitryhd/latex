%%%%%%%%%%%%%%%%%%%%%%%%%%%%%%%%%%%%%%%%%
% Stylish Curriculum Vitae
% LaTeX Template
% Version 1.0 (18/7/12)
%
% This template has been downloaded from:
% http://www.LaTeXTemplates.com
%
% Original author:
% Stefano (http://stefano.italians.nl/)
%
% IMPORTANT: THIS TEMPLATE NEEDS TO BE COMPILED WITH XeLaTeX
%
% License:
% CC BY-NC-SA 3.0 (http://creativecommons.org/licenses/by-nc-sa/3.0/)
%
% The main font used in this template, Adobe Garamond Pro, does not 
% come with Windows by default. You will need to download it in
% order to get an output as in the preview PDF. Otherwise, change this 
% font to one that does come with Windows or comment out the font line 
% to use the default LaTeX font.
%
%%%%%%%%%%%%%%%%%%%%%%%%%%%%%%%%%%%%%%%%%

\documentclass[a4paper, oneside, final]{scrartcl} % Paper options using the scrartcl class

\usepackage{scrpage2} % Provides headers and footers configuration
\usepackage{titlesec} % Allows creating custom \section's
\usepackage{marvosym} % Allows the use of symbols
\usepackage{tabularx,colortbl} % Advanced table configurations
\usepackage{fontspec} % Allows font customization
\usepackage{color}
\usepackage[usenames,dvipsnames]{xcolor}

\usepackage{polyglossia}
\setmainlanguage{russian} 
\setotherlanguage{english}
\newfontfamily\russianfont[Script=Cyrillic]{Tahoma}

\defaultfontfeatures{Mapping=tex-text}

\titleformat{\section}{\large\scshape\raggedright}{}{0em}{}[\titlerule] % Section formatting

\pagestyle{scrheadings} % Print the headers and footers on all pages

\addtolength{\voffset}{-2.3cm} % Adjust the vertical offset - less whitespace at the top of the page
\addtolength{\textheight}{5.7cm} % Adjust the text height - less whitespace at the bottom of the page

%\definecolor{light-gray}{gray}{0.95}
\newcommand{\gray}{\rowcolor[gray]{.93}} % Custom highlighting for the work experience and education sections

%----------------------------------------------------------------------------------------
% FOOTER SECTION
%----------------------------------------------------------------------------------------

\renewcommand{\headfont}{\normalfont\rmfamily\scshape} % Font settings for footer

\cofoot{
\addfontfeature{LetterSpace=10.0}\fontsize{12.5}{12}\selectfont % Letter spacing and font size


{\Letter} kseniya.terekhina@phystech.edu \ {\Large\Telefon} (916) 467-78-86 % Your email address and phone number
}

%----------------------------------------------------------------------------------------

\begin{document}

\begin{center} % Center everything in the document

%----------------------------------------------------------------------------------------
% ÊHEADER SECTION
%----------------------------------------------------------------------------------------

{\addfontfeature{LetterSpace=5.0}\fontsize{24}{24}\selectfont\scshape Ксения Терехина} % Your name at the top



%----------------------------------------------------------------------------------------
%	WORK EXPERIENCE
%----------------------------------------------------------------------------------------

\section{\textbf{\textcolor{BlueViolet}{Образование}}}

\begin{tabularx}{0.99\linewidth}{cp{0.8\linewidth}l}
\gray \textbf{2014 - 2016} & \textbf{Московский физико-технический институт (МФТИ ГУ)} {Магистр. Cистемный анализ и управление в больших системах} \\
\gray & \quad \\
\gray & \textbf{Российская академия народного хозяйства и государственной службы при Президенте РФ (РАНХиГС) } \\ 
\gray & Магистр. Финансы и Экономика \\
 
\end{tabularx}

\vspace{10pt}
\begin{tabularx}{0.99\linewidth}{cp{0.8\linewidth}l}
\gray \textbf{2010 - 2014} & \textbf{Московский физико-технический институт (МФТИ ГУ)} {Бакалавр. Cистемный анализ и управление } \\
\gray  & \quad \\
\gray  & \textbf{Российская академия народного хозяйства и государственной службы при Президенте РФ (РАНХиГС) } \\
\gray  & {Бакалавр. Экономика}\\
 
\end{tabularx}

%----------------------------------------------------------------------------------------
%	EDUCATION
%----------------------------------------------------------------------------------------

\section{\textbf{\textcolor{BlueViolet}{Опыт работы}}}
\begin{tabularx}{0.99\linewidth}{cp{0.8\linewidth}l}
\gray \textbf{2014 - н/вр} & \textbf{Институт прикладных экономических исследований при РАНХиГС} \\
\gray & Младший научный сотрудник \\
%& Extra information about degree
\end{tabularx}

%----------------------------------------------------------------------------------------
%	SKILLS
%----------------------------------------------------------------------------------------


\section{\textbf{\textcolor{BlueViolet}{Навыки}}}
\begin{tabular}{ @{} >{\bfseries}l @{\hspace{6ex}} l }
Языки & Английский (Технический)\\ & Испанский (Базовые знания)\\
Языки програмирования & Python, Matlab/Octave, Wolfram Mathematica \\
Статические пакеты  & Stata, Eviews \\
Прикладные программы & MS Word, Excel, OS Linux, Git, Vim, LaTeX \\

\end{tabular}

%----------------------------------------------------------------------------------------


\section{\textbf{\textcolor{BlueViolet}{Участие в конференциях}}}
\begin{minipage}{.99\linewidth}
\begin{itemize}
\item INFOS2014, Польша, Сентябрь 2014 
\item 57-я научная конференция МФТИ, Москва, Ноябрь 2014 
\item Гайдаровский форум 2015 "Современная экономика: теория, политика, инновации глазами молодых ученых", Москва, Январь 2015 
\end{itemize}

\section{\textbf{\textcolor{BlueViolet}{Публикации}}}
\begin{itemize}
\item  ITHEA® 2014, Rzeszow, Poland; Sofia, Bulgaria/ Transactions on Business and Engineering Intelligent Applications/ " Building Noise Immunity Models for GDP Forecast Based on Electrical Power Consumption" Ksenia Terekhina, Mikhail Alexandrov, Oleksiy Koshulko
\end{itemize}

\end{minipage}

\section{\textbf{\textcolor{BlueViolet}{Сертификаты}}}

\begin{tabular}{cp{11cm}p{1cm}}
2013 & \textbf{Machine Learning by Andrew Ng} & Coursera  \\
2014 & \textbf{Statistical Learning} & Stanford University  \\
2015 & \textbf{Introduction to Computer Science and Programming Using Python} & Edx  \\

\end{tabular}

%---------------------------------------------------------------------------------

\end{center}

\end{document}
